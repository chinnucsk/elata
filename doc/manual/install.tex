%%
%%   Copyright (c) 2011, Nokia Corporation
%%   All Rights Reserved.
%%
%%   The contents of this file are subject to the 3-clause BSD License,
%%   (the "License"); you may not use this file except in compliance 
%%   with the License. You should have received a copy of the 3-clause
%%   BSD Licensee along with this software. If not, it can be
%%   retrieved online at http://www.opensource.org/licenses/BSD-3-Clause.
%%
%%   Software distributed under the License is distributed on an "AS IS"
%%   basis, WITHOUT WARRANTY OF ANY KIND, either express or implied. See
%%   the License for the specific language governing rights and limitations
%%   under the License.
%%
%% @author Dmitry Kolesnikov <dmitry.kolesnikov@nokia.com>
%%
\section{Getting started with ELATA}

Project delivery is available at http://github.com/fogfish/elata; and is composed from:
\begin{itemize}
\item agent (agt) -- gathers telemetry information about software components, performs latency measurements, etc;
\item back-end (be) -- persist and analyze telemetry data received from agent;
\item front-end (fe) -- hosts web application and implements REST interface to access telemetry.
\item common components such as Key-Value Store Interface (kvs), WebMachine API wrapper (rest) and Time Series Analysis (tsa).   
\end{itemize}


\subsection{How to install ELATA agent}

\subsubsection{ELATA agent dependencies}

Erlang/OTP run-time, recommended version R14B02
\begin{verbatim}
   curl -O http://www.erlang.org/download/otp_src_R14B02.tar.gz
   tar -zxvf otp_src_R14B02.tar.gz
   cd otp_src_R14B02
   ./configure --prefix=/usr/local/otp \
               --enable-smp-support    \
               --enable-kernel-poll    \
               --enable-hipe           \
               --disable-java
   make && make install            
\end{verbatim}

WebMachine, recommended version 1.8.1 \\
Note: WebMachine Makefile needs to be patched, patch is delivered as part of ELATA project
\begin{verbatim}
   git clone https://github.com/basho/webmachine
   cd webmachine
   patch Makefile < webmachine-Makefile.path
   make && make install DESTDIR=/usr/local/otp
\end{verbatim}


\subsubsection{Build ELATA agent}

Current release of ELATA contains only source code. Therefore, ELATA agent needs to be build once before it is massively deployed across world.    

\begin{verbatim}
   git clone https://github.com/fogfish/elata
   cd elata
   ./configure --with-erlang=/usr/local/otp
   make
\end{verbatim}  

As the result ./elata/agt/elata\_agt-x.y.z.tgz is build. The file contains standalone binary (including Erlang/OTP) ready to be distributed and installed to network components (architecture and OS should be same).


To install and run the agent:
\begin{verbatim}
   tar -zxvf elata_agt-x.y.z.tgz
   ./bin/elata_agt
\end{verbatim}


\subsubsection{ELATA agent configuration}
The agent behavior is configured
\begin{itemize}
\item at build-time via ./elata/agt/priv/sys.config.in
\item at run-time via  /lib/elata\_agt-x.y.z/priv/sys.config.
\end{itemize}
Please refer to Erlang documentation (http://erlang.org) for details of configuration file format.


The following options could be changed to modify agent behavior: 
\begin{itemize}
\item elata\_agt/port (default 8080), TCP/IP port used for communication with back-end.
\item elata\_agt/worker, defines behavior of measurement agent
\subitem idletime (default 300000 / 5 min), time in microseconds to wait between measurement trials. 
\subitem thinktime (default 1000 / 1 sec), time in microsecond to wait between consequent measurement.
\subitem cycle (default 10), number of consequent measurement cycles perform during a trial. Agent reports aggregated statistic based this measurements. 
\end{itemize}


\subsection{How to install ELATA back-end/front-end}


\subsubsection{ELATA back-end/front-end dependencies}

Please refer to Chapter -- ELATA agent dependencies for installation instruction of Erlang/OTP runtime and WebMachine.


RRD Tool version 1.4.5 is used as storage and graph rendering back-end. Please refer to http://www.mrtg.org/rrdtool/ for details. 


Optionally yuicompressor is used at build procedure to minimize output of JavaScript and Style sheet. Please refer to http://yuilibrary.com/downloads for details.


\subsubsection{Build ELATA front-end/back-end}

Current release of ELATA contains only source code. Therefore, build procedure is required before it is deployed to target hardware.    

\begin{verbatim}
   git clone https://github.com/fogfish/elata
   cd elata
   ./configure \
      --with-erlang=/usr/local/otp \
      --with-yuicompressor=/path/to/yuicompressor-x.y.z.jar
   make
\end{verbatim}  

As the result ./elata/fe/elata\_fe-x.y.z.tgz is build. The file contains standalone binary (including Erlang/OTP) ready to be distributed and installed to network components (architecture and OS should be same).


To install and run the agent:
\begin{verbatim}
   tar -zxvf elata_fe-x.y.z.tgz
   ./bin/elata_fe
\end{verbatim}


\subsection{ELATA back-end configuration}

List of connected agents needs to be explicitly defined
\begin{itemize}
\item at build-time via ./elata/be/priv/agent.config.in
\item at run-time via  /lib/elata\_be-x.y.z/priv/agent.config.
\end{itemize}

Each agent node available in the system needs to be defined by the record in following format: 
\begin{verbatim}
{kvset, <<"rd-local">>,             %% unique agent identity
    [
       {title,      <<"RD/Local">>}, %% human readable name
       {host,  {"127.0.0.1", 8080}}, %% host, port to communicate with
       {sync,                60000}  %% telemetry synchronizations timeout
      %{proxy, {"127.0.0.1", 8080}}  %% proxy host, port 
    ]
}.
\end{verbatim}


The application behavior is configured
\begin{itemize}
\item at build-time via ./elata/fe/priv/sys.config.in
\item at run-time via  /lib/elata\_fe-x.y.z/priv/sys.config.
\end{itemize}


The following options could be changed to modify agent behavior: 
\begin{itemize}
\item elata\_be/codepath (default /usr/local/macports), path prefix to external tools.
\item elata\_be/datapath (default /private/tmp/elata), folder to persist telemetry data
\item elata\_fe/port (default 8000), TCP/IP port used for communication with client application (web browser)
\item elata\_fe/datapath (default /private/tmp/elata), folder to persist user profiles and telemetry data. Note: in current release elata\_fe/datapath and elata\_be/datapath MUST point to same location.
\end{itemize}



